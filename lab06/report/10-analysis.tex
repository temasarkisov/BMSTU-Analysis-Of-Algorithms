\chapter{Аналитическая часть}

\section{Задача коммивояжера}

Коммивояжёр (фр. {\ttfamily commis voyageur}) — бродячий торговец. Задача коммивояжёра — важная задача транспортной логистики, отрасли, занимающейся планированием транспортных перевозок. Коммивояжёру, чтобы распродать нужные и не очень нужные в хозяйстве товары, следует объехать n пунктов и в конце концов вернуться в исходный пункт. Требуется определить наиболее выгодный маршрут объезда. В качестве меры выгодности маршрута (точнее говоря, невыгодности) может служить суммарное время в пути, суммарная стоимость дороги, или, в простейшем случае, длина маршрута \cite{commi2}.


\section{Решение полным перебором}

Наиболее идейно простым алгоритмом решения задачи коммивояжера \cite{commi} является полный перебор решений с выбором кратчайшего из полученных путей. Очевидным недостатком данного алгоритма является необходимость перебора значительного числа комбинаций, которое с ростом числа городов быстро выходит за рамки вычислительных мощностей современных компьютеров. Трудоёмкость алгоритма полного перебора -- $O(n!)$.

\section{Муравьиные алгоритмы}

Идея муравьиного алгоритма -- моделирование поведения муравьев, связанное с их способностью быстро находить кратчайшие пути и адаптироваться к изменяющимся условиям, т.е. искать новые кратчайшие пути \cite{ant1}. При своём движении муравей метит свой путь феромоном, и эта информация используется прочими для выбора пути. Таким образом, более короткие пути будут сильнее обогащаться феромоном и станут более предпочтительны для всей колонии. С помощью моделирования испарение феромона, т.е. отрицательной обратной связи, гарантируется, что найденное локально оптимальное решение не будет единственным -- будут предприняты попытки поиска других путей.


Опишем правила поведения муравья применительно к решению задачи коммивояжера \cite{shtovba}:

\begin{itemize}

	\item муравьи имеют 'память' - запоминают уже посещенные города, чтобы избегать повторений: обозначим через $J_{i,k}$ список городов, которые необходимо пройти муравью $k$, находящемуся в городе $i$;

	\item муравьи обладают 'зрением' - видимость есть эвристическое желание посетить город $j$, если муравей находится в городе $i$. Уместно считать видимость обратно пропорциональной расстоянию между соответствующими городами $\eta_{i,j} = 1/D_{i,j}$;

	\item муравьи обладают 'обонянием' -- могут улавливать след феромона, подтверждающий желание посетить город $j$ из города $i$ на основании опыта других муравьев. Обозначим количество феромона на ребре $(i,j)$ в момент времени $t$ через $\tau_{i,j}(t)$. 

\end{itemize}

Вероятность перехода из вершины $i$ в вершину $j$ определяется по следующей формуле \ref{form:way}\\   

\begin{equation}\label{form:way} 
	p_{i,j}={\frac {(\tau_{i,j}^{\alpha })(\eta_{i,j}^{\beta })}{\sum (\tau_{i,j}^{\alpha})(\eta_{i,j}^{\beta })}}
\end{equation}

где \quad$ \tau_{i,j} - $ количество феромонов на ребре i до j;


$\eta_{i,j} - $ эвристическое расстояние от i до j;


$\alpha - $ параметр влияния феромона;


$\beta - $ параметр влияния расстояния.

Пройдя ребро $(i,j)$ , муравей откладывает на нём некоторое количество феромона, которое должно быть связано с оптимальностью сделанного выбора. Пусть $T _{k} (t)$ есть маршрут, пройденный муравьём $k$ к моменту времени $t$ , $L _{k} (t)$ - длина этого маршрута, а $Q$ - параметр, имеющий значение порядка длины оптимального пути. Тогда откладываемое количество феромона может быть задано в виде:

\begin{equation}\label{form:add} 
	{\displaystyle \Delta \tau _{i,j}^k={\begin{cases}Q/L_{k}& {\mbox{Если k-ый мурваей прошел по ребру ij;}}\\0&{\mbox{Иначе}}\end{cases}}}
\end{equation}

где \quad Q - количество феромона, переносимого муравьём;


Тогда

\begin{equation}\label{form:add1} 
	\Delta \tau _{i,j}= \tau _{i,j}^0 + \tau _{i,j}^1 + ... + \tau _{i,j}^k
\end{equation}


На основе приведённого выше описания алгоритма можно оценить его трудоемкость: $O(t_{max} \cdot m \cdot n^2)$, где $t_{max}$ - число итераций алгоритма ('время жизни колонии'), $m$ - число муравьёв, $n$ - число городов.



\section*{Вывод}
В данном разделе были рассмотрены общие принципы муравьиного алгоритма и применение его к задаче коммивояжёра. 
